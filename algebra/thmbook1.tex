\documentclass[11pt, a4paper]{article}
\usepackage[theorems,macros]{styles}
\usepackage[
nomarginpar,
headheight=1in
]{geometry}

% --- Header and Footer Setup ---
\usepackage{fancyhdr}
\pagestyle{fancy}
\fancyhead{} % clear all header fields
\fancyhead[RO]{MATH 80.1}
\fancyhead[LO]{Thmbook 1}
\fancyfoot{} % clear all footer fields
\fancyfoot[RO]{\thepage}
\fancyfoot[C]{\leftmark}
% \renewcommand{\headrulewidth}{0.4pt}
% \renewcommand{\footrulewidth}{0pt}
\setlength{\headsep}{1em}
% --------------------------------------------- %
\usepackage{tikz-cd}

\newcommand{\cyc}[1]{\langle {#1} \rangle}
\newcommand{\lcm}{\mathop{\text{lcm}}}
\newcommand{\sgn}{\mathop{\text{sgn}}}
\title{}
\author{Lynus Kho}

\allowdisplaybreaks

\begin{document}
\section{Recall}
\begin{theorem}[Bezout's Identity]
  Let $\text{gcd}(a,b) = d$. Then there exist integers $x$ and $y$ such that $ax+by =d$. Moreover, integers of the form $as+bt$ (i.e. linear combinations of $a,b$) are exactly the multiples of $d$. 
\end{theorem}
This theorem gives rise to an extremely important corollary:
\begin{corollary}\label{cor:1}
  Integers $a$ and $b$ are relatively prime (i.e. $\text{gcd}(a,b) = 1$) iff there exist $s,t\in \mathbb{Z}$ such that $as+bt=1$.
\end{corollary}
\begin{theorem}
  Let $n,a$ be positive integers and let $d = \gcd(a,n)$. Then the equivalence
  \begin{equation*}
    ax \equiv 1 \pmod{n}
  \end{equation*}
  has a solution if and only if $d=1$.
\end{theorem}
\begin{proof}
  We have
  \begin{equation*}
    ax \equiv 1 \pmod{n} \iff \exists k\in \Z,\;  ax = kn + 1 \iff \gcd(a,n)=1.
  \end{equation*}
\end{proof}
\begin{remark}
  We can consider $x$ as the inverse of $a$ under $U(n)$. Thus we can restate this as ``$a$ has an inverse $x$ iff $\gcd(a,n)=1$.'' This theorem justifies why the only members of $U(n)$ are those coprime to $n$.
\end{remark}
\begin{definition}[Euler's Totient Function]
  Given a positive integer $n$, the function $\phi(n)$ counts the positive integers less than $n$ that are relatively prime to $n$. Formally,
  \begin{equation*}
  \phi(n) = \left|\left\{1\leq k \leq n \mid \text{gcd}(k,n)=1\right\}\right|.
  \end{equation*}
\end{definition}

\begin{theorem}
  Euler's Phi is a multiplicative function. That is, if $\gcd(m,n) = 1$ then $\phi(mn) = \phi(m) \phi(n)$.
\end{theorem}
\begin{theorem}\label{thm:phiprimepower}
  Let $p$ be prime. Then $\phi (p^{n}) = p^{n} - p^{n-1}$.
\end{theorem}

\section{Groups}
\begin{definition}
    Let $G$ be a set. A \textit{binary operation} on $G$ is a function that assigns each ordered pair of elements of $G$ to an element of $G$.
\end{definition}
\begin{definition}
  Let $G$ be a set together with a binary operation that assigns to each ordered pair $(a,b)$ of elements of $G$ to an element of $G$ denoted by $ab$. We say $G$ is a \textit{group} under this operation if the following three properties are satisfied.
  \begin{enumerate}

    \item \textit{Associativity}. We have $(ab)c = a(bc)$ for all $a,b,c$ in $G$.
    \item \textit{Identity}. There is an element $e$ (called the \textit{identity}) in $G$ such that $ae=ea=a$ for all $a$ in $G$.
    \item \textit{Inverses}. For each element $a$ in $G$, there is an element $b$ in $G$ (called an \textit{inverse} of $a$) such that $ab=ba=e$.

  \end{enumerate}
\end{definition}
\begin{example}
    The following are all examples of groups:
    \begin{enumerate}

        \item The sets $\Z, \Q, \R, \mathbb{C}$ are all groups under addition. In all cases the identity is $0$ and the inverse of $a$ is $-a$.
        \item The set of positive rationals $\Q^{+}_{\times}$ is a group under ordinary multipication.
        \item The set $\Z_n = \{0,1,\dots,n-1\}$ is a group under addition modulo $n$.
        \item The set of $2 \times 2$ nonsingular matrices is called the \textit{general linear group} of $2 \times 2$ matrices over $\R$, denoted $\text{GL}(2,\R)$. This group is non-commutative.
        \item Let $\mathbb{F}$ be a field, such as $\R, \C,$ or $\Z_p$. Then $\text{GL}(n,\mathbb{F})$ is a group for any positive integer $n\geq 1$, prime .
        \item For any positive integer $n$, we define $U(n)$ to be the set of all positive integers less than $n$ and relatively prime to $n$. The operation is multiplication modulo $n$. This group is simply known as the \textit{multiplicative group of integers modulo} $n$.
        \item Notice that in $U(8) = \Set{1,3,5,7}$, we have the property $3\cdot 5 = 7$, $5\cdot 7 = 3$, and $7 \cdot 3 = 5$. Thus $U(8)$ exhibits the properties of (i.e. is isomorphic to) the Klein-4 group.
        \item Consider the symmetries of a regular $n$-gon with $n\geq 3$. The corresponding group is denoted $D_n$ and is called the \textit{dihedral group of order} $2n$. For instance, a square has symmetries $D_4 = \{R_{0},R_{90},R_{180},R_{270},H,V,D,D'\}$ and is called the dihedral group of order $8$. This group is non-commutative.

    \end{enumerate}
\end{example}
\begin{definition}
    A group with the property that $ab=ba$ for every pair of elements $a,b\in G$ is said to be \textit{Abelian}.
\end{definition}
\begin{theorem}
    In a group, there is only one identity element.
\end{theorem}
\begin{proof}
    Let $e,e'$ be identities in $G$. Then
    \begin{enumerate}

        \item $ae=a$ for all $a\in G$, and
        \item $e'b=b$ for all $b\in G$.

          Setting $a:= e'$ and $b:= e$ yields $e'e = e'$ and $e'e = e$, respectively, which proves the claim.
    \end{enumerate}
\end{proof}

\begin{theorem}
    In a group, right and left cancellation hold; that is, $ba=ca$ implies $b=c$, and $ab=ac$ implies $b=c$
\end{theorem}
\begin{theorem}
    For each element $a$ in a group $G$, there is a unique element $a^{-1}$ in $G$ such that $aa^{-1} = a^{-1}a = e$.
\end{theorem}

\begin{definition}
    Let $g$ be an element of a group $G$. If $n$ is positive, then we define
    \begin{equation*}
      g^{n} := \underbrace{gg\dots g}_{\text{$n$ factors}}.
    \end{equation*}
    if $n=0$, then $g^{0}:= e$, and if $n$ is negative,
    \begin{equation*}
      g^{n} := (g^{-1})^{|n|}.
    \end{equation*}
\end{definition}
\begin{example}
  For $(ab)^{-2}$ we have
  \begin{equation*}
    (ab)^{-2} = [(ab)^{-1}]^{2} = (b^{-1}a^{-1})^{2} = b^{-1}a^{-1}b^{-1}a^{-1}.
  \end{equation*}
  using the socks-and-shoes principle described later.
\end{example}
\begin{remark}
    The order in which the $-1$ and the $|n|$ appears is not so important, since for positive $n$,
    \begin{equation*}
      g^{-n} = (g^{-1})^{n}
    \end{equation*}
    but clearly
    \begin{equation*}
      (g^{-1})^{n} g^{n} = e
    \end{equation*}
    and so 
    \begin{equation*}
      (g^{-1})^{n} = (g^{n})^{-1}.
    \end{equation*}
    
\end{remark}
\begin{theorem}
    The laws of exponents hold, that is, for integers $m,n$ and any group element $g$, we have
    \begin{enumerate}

        \item $g^{m}g^{n} = g^{m+n}$
        \item $(g^{m})^{n} = g^{mn}$.

    \end{enumerate}
\end{theorem}
\begin{remark}
  Note that in general, we \textbf{do not} have $(ab)^{n} = a^{n}b^{n}$. It is the case that
  \begin{equation*}
    (ab)^{n} = \underbrace{abab\dots ab}_{\text{$n$ times}}.
  \end{equation*}
\end{remark}
\begin{theorem}[Socks and Shoes]
  For group elements $a$ and $b$, $(ab)^{-1} = b^{-1}a^{-1}$.
\end{theorem}
\begin{theorem}[Pants, Socks and Shoes]
  For group elements $a,b,c$ we have $(abc)^{-1} = c^{-1}b^{-1}a^{-1}$.
\end{theorem}

\begin{definition}[Cayley Table]
  A \textit{Cayley table} for a group $G$ with $n$ elements is an $n \times n$ array where a row corresponds to an element $a$ in $G$, a column corresponds to an element $b$ in $G$, and the entry in the $a$-row and $b$-column is the product $ab$. 
\end{definition}
\begin{theorem}
    In a Cayley table, each element in a group occurs exactly once in each row and each column.
\end{theorem}
\begin{proof}
  If an element is in the $a$-row, then it is of the form $ax$ for some $x\in G$. Suppose $ax = ay$, i.e. the entry in $(a,x)$ is equal to the entry in $(a,y)$. Then by cancellation, $x=y$, so every entry in the row is unique. The same argument mutatis mutandis can be used to prove the uniqueness for columns.
\end{proof}
\begin{theorem}
    Consider the Cayley table of $G$ as an $n \times n$ matrix. Then $G$ is Abelian iff its Cayley table is symmetric.
\end{theorem}

\section{Finite Groups and Subgroups}
\begin{definition}
    The number of elements $|G|$ of a group $G$ is called its \textit{order}.
\end{definition}
\begin{definition}
    The \textit{order} of an element $g$ in a group $G$ is the smallest positive integer $n$ such that $g^{n} = e$. If no such integer exists, we set $g^{n} := \infty$. We denote the order of $g$ as $|g|$.
\end{definition}

\begin{definition}
    If a subset $H$ of a group $G$ is itself a group under the operation of $G$, we say that $H$ is a \textit{subgroup} of $G$.
\end{definition}
\begin{definition}
    If $H$ is a subgroup of $G$ but $H\neq G$, we say that $H$ is a \textit{proper subgroup of $G$}.
\end{definition}
\begin{definition}
    The lattice of subgroups of a group $G$ is the lattice whose elements are the subgroups of $G$, with the partial ordering being set inclusion $\subseteq$.
\end{definition}
\begin{example}
  For $U(15)$ we have the following lattice of subgroups:
  \begin{figure}[htb!]
    \centering
    \begin{tikzcd}[every arrow/.append style={dash}]
      &&U(15) \ar[d]\ar[dl]\ar[dr]\\
      & \Set{1,2,4,8} \ar[dr] & \Set{1,4,11,14} \ar[dl] \ar[d] \ar[dr] &\Set{1,4,7,13}\ar[dl]\\
      & \Set{1,11} \ar[dr] & \Set{1,4} \ar[d] & \Set{1,14} \ar[dl]\\
      && \Set{1}
    \end{tikzcd}
    \label{fig:}
  \end{figure}
\end{example}
\begin{theorem}[One-Step Subgroup Test]
    Let $G$ be a group. Then a subset $H$ is a subgroup of $G$ if and only if the following hold:
    \begin{enumerate}

        \item $H$ is nonempty
        \item If $a,b\in H$ then $ab^{-1}\in H$.

    \end{enumerate}
\end{theorem}
\begin{proof}
    Since the operation of $H$ is the same as in $G$, it is clear that this operation is associative. Now we show that $e\in H$. Since $H$ is nonempty, we can speak of an element $a\in H$. By premise, since $a\in H$ we know $aa^{-1}=e\in H$. Now let $x\in H$; we wish to show that $x^{-1}\in H$ as well. Take $a = e$ and $b= x$ in the premise: then $eb^{-1} = b^{-1} \in H$ as desired.
\end{proof}

\begin{theorem}[Two-Step Subgroup Test]
    Let $G$ be a group. Then a subset $H$ is a subgroup of $G$ if and only if the following hold:
    \begin{enumerate}

        \item $H$ is nonempty.
        \item If $a,b\in H$ then $ab\in H$.
        \item If $a\in H$ then $a^{-1} \in H$.


    \end{enumerate}
\end{theorem}
\begin{proof}
  Notice we just need to show that $e\in H$. But this follows since $H$ is nonempty and $aa^{-1}=e\in H$ by (2) and (3).
\end{proof}
\begin{theorem}[Finite Subgroup Test]
    Let $H$ be a nonempty finite subset of a group $G$. If $H$ is closed under the operation of $G$, then $H$ is a subgroup of $G$.
\end{theorem}
\begin{proof}
  By the Two-Step Subgroup Test, we only need to prove (3), that $a^{-1}\in H$ whenever $a\in H$. If $a = e$, we are done, so suppose $a\neq e$ and consider the sequence
  \begin{equation*}
    a,a^{2},\dots.
  \end{equation*}
  By closure, all of these elements belong to $H$, but since $H$ is finite, not all of these elements are distinct. Say $a^{i} = a^{j}$ and WLOG $i>j$. Then $a^{i-j} = e$ and since $a\neq e$ we have $i-j > 1$. Therefore
  \begin{equation*}
      aa^{i-j-1} = a^{i-j}=e \implies a^{i-j-1} = a^{-1}.
  \end{equation*}
  But $i-j-1\geq 0$ implies $a^{i-j-1}\in H$ and we are done.
\end{proof}

\begin{definition}
  A group $H$ is called \textit{cyclic} iff there is an element $a\in H$ such that $H = \{a^{n} \mid n\in \Z\}$. Such an element $a$ is called a \textit{generator} of $H$.

  Let $G$ be a group and $a\in G$ a group element. We call $\langle a \rangle$ the \textit{cyclic subgroup of $G$ generated by $a$} defined by
  \begin{align*}
    \langle a \rangle :=& \Set{a^{n} \given n\in \Z} \\
    =& \{ \dots, a^{-2},a^{-1},e,a,a^{2},\dots\}
  \end{align*}
\end{definition}

\begin{theorem}
    For any $a\in G$, $\langle a \rangle$ is a subgroup of $G$.
\end{theorem}
\begin{proof}
    We use the One-Step Subgroup Test. Clearly $a\in \langle a\rangle$, so it is nonempty. Let $a^{i},a^{j} \in \langle a \rangle$. Then $a^{i}a^{-j} = a^{i-j} \in \langle a \rangle$ by laws of exponents.
\end{proof}
\begin{definition}
    Let $S$ be a collection of elements from a group $G$. Then  we call $\langle S \rangle$ the \textit{subgroup generated by $S$}, defined as the smallest subgroup of $G$ containing $S$. More precisely, $\langle S \rangle$ is the subgroup with the property that $S\subseteq \langle S \rangle $, and if $S\subseteq H$, then $\langle S \rangle \subseteq H$.
\end{definition}
\begin{remark}
  In linear algebra, this is simply the span.
\end{remark}
\begin{definition}
  The center $Z(G)$ of a group $G$ is the subset of elements in $G$ that commute with every element of $G$, that is,
  \begin{equation*}
    Z(G) := \Set{ a\in G \given ax = xa \text{ for all }x\in G}.
  \end{equation*}
\end{definition} 
\begin{theorem}
    The center of a group $G$ is a subgroup of $G$.
\end{theorem}
\begin{proof}
  We use the One-Step Subgroup Test. First, we have $ex=xe=x$ for all $x\in G$, so $e\in Z(G)$. Now let $a,b\in Z(G)$. Then for any $x\in G$
  \begin{align*}
    (ab^{-1})x &= b^{-1}(ax) \\
    &= b^{-1}(ax)bb^{-1} \\
    &= b^{-1}b(ax)b^{-1} \\
    &= axb^{-1} \\
    &= xab^{-1}
  \end{align*}
  and so $ab^{-1}\in Z(G)$, as desired.
\end{proof}
\begin{definition}
  Let $a$ be a fixed element of a group $G$. The \textit{centralizer of $a$ in $G$}, denoted $C(a)$, is the set of all elements in $G$ that commute with $a$. In other words,
  \begin{equation*}
    C(a) := \Set{g\in G \given ga=ag}.
  \end{equation*}
\end{definition}
\begin{theorem}
  For each $a\in G$, $C(a)$ is a subgroup of $G$.
\end{theorem}
\begin{proof}
    We use the same proof as above, modified slightly.
\end{proof}

\subsection{Some Useful Theorems}
\begin{theorem}
  Let $a,x$ be any group elements of $G$ and $n\in \Z$. Then we have
  \begin{equation*}
  (xax^{-1})^{n} = xa^{n}x^{-1}
  \end{equation*}
\end{theorem}

\section{Cyclic Groups}
We have already defined cyclic groups in the previous section. Here are some of their properties.
\begin{theorem}\label{thm:criterion_for_ai_eq_aj}
  Let $a\in G$. If $|a| = \infty$, then $a^{i} = a^{j}$ iff $i=j$. If $|a|=n\in \Z^{+}$, then $\cyc{a} = \Set{e,a,a^{2},\dots,a^{n-1}}$ and $a^{i}=a^{j}$ iff $n \mid i-j$.
\end{theorem}
\begin{proof}
    Suppose first that $|a| = \infty$. Then if $a^{i} = a^{j}$, we must have $a^{i-j} = e$. Since $a$ has infinite order, it must be that $i=j$.

    Now suppose $|a|=n < \infty$. That $\cyc{a} = \Set{e,a,a^{2},\dots, a^{n-1}}$ can be proven easily with the division algorithm. Now suppose $a^{i} = a^{j}$; we wish to show that $n \mid i-j$. Observe that $a^{i-j} = e$ and by the division algorithm, we can find integers $q,r$ such that
    \begin{equation*}
        i-j = nq + r, \qquad 0\leq r < n.
    \end{equation*}
    Thus $a^{i-j} = a^{nq+r} = a^{r}$ But $a^{i-j} = e$ and since $r<n$, it must be that $r = 0$ (otherwise $n$ is not the smallest integer such that $a^{n} = e$). Therefore, $i-j = nq$, that is, $n$ divides $i-j$.
\end{proof}
\begin{corollary}\label{cor:ak_eq_e_iff_ord_div_k}
    We have $a^{k} = e$ if and only if $|a| \mid k$.
\end{corollary}
\begin{corollary}
  We have $|\cyc{a^{k}}| = |a^{k}|$ for any integer $k$.
\end{corollary}
\begin{corollary}
    If $a,b\in G$ where $|G|< \infty$ and $ab=ba$, then $|ab|$ divides $|a||b|$.
\end{corollary}
\begin{proof}
  Let $|a| = m$ and $|b| = n$. Then $(ab)^{mn} = (a^{m})^{n}(b^{n})^{m} = e$ and so by the previous corollary, $|ab| \mid mn$.
\end{proof}
\begin{remark}
  There is essentially only one cyclic group of each order. If $|a| = \infty$ then $\cyc{a} \cong \Z$, and if $|a| = n < \infty$ then $\cyc{a} \cong \Z_n$.
\end{remark}

\begin{theorem}\label{thm:ord_cyc_sub}
  Let $|a| = n$. Then $\cyc{a^{k}} = \cyc{a^{\gcd(n,k)}}$ and $|a^{k}| = n / \gcd (n,k)$ for any integer $k$.
\end{theorem}
\begin{proof}
  Write $d = \gcd (n,k)$ and let $k = dr$ for some integer $r$. Since $a^{k} = (a^{d})^{r}$, we have $\cyc{a^{k}} \subseteq \cyc{a^{d}}$ by closure. By \autoref{cor:1} we can find integers $s,t\in \Z$ such that $d = ns + kt$. So,
  \begin{equation*}
    a^{d} = a^{ns + kt} = ea^{kt} = (a^{k})^{t} \in \cyc{a^{k}}
  \end{equation*}
  which shows that $\cyc{a^{d}}\subseteq \cyc{a^{k}}$. Therefore $\cyc{a^{\gcd(n,k)}} = \cyc{a^{k}}$.

  For the second part, notice that
  \begin{align*}
    |a^{k}| &= \min \Set{m\in \Z^{+} \given (a^{k})^{m} = e } \\
    &= \min \Set{m \in \Z^{+} \given n \mid km} \\
    &= \min \Set{m\in \Z^{+} \given km\text{ is a multiple of } n}
  \end{align*}
  and so $k |a^{k}| = \text{lcm}(n,k)$. Multiplying both sides by $\gcd(n,k)$ gives
  \begin{align*}
    k\gcd(n,k) |a^{k}| &= \gcd(n,k) \text{lcm}(n,k) \\
    &= nk
  \end{align*}
  and so $|a^{k}| = n / \gcd(n,k)$, as desired.
\end{proof}

\begin{corollary}\label{cor:ord_sub_div_ord_grp}
  In a finite cyclic group $\cyc{a}$, the order of an element divides the order of the group. In other words, if $|a| = n$ and $|a^{k}|=m$ then $m \mid n$.
\end{corollary}
\begin{corollary}
  Let $|a| = n$. Then $\cyc{a^{i}} = \cyc{a^{j}}$ iff $\gcd(n,i) = \gcd(n,j)$.
\end{corollary}
\begin{corollary}\label{cor:generators_of_grp}
  Let $|a|= n$. Then $\cyc{a} = \cyc{a^{j}}$ iff $\gcd(n,j) = 1$.
\end{corollary}
\begin{corollary}
  An integer $k$ in $\Z_n$ is a generator of $\Z_n$ iff $\gcd(n,k) = 1$.
\end{corollary}

\subsection{Classification of Subgroups of Cyclic Groups}
\begin{theorem}[Fundamental Theorem of Cyclic Groups]\label{thm:fund_cyc}
  Every subgroup of a cyclic group is cyclic. Moreover, if $|a| =n$, then the order of any subgroup of $\cyc{a}$ is a divisor of $n$, and for each positive divisor $d$ of $n$, the group $\cyc{a}$ has exactly one subgroup of order $d$ --- namely $\cyc{a^{n/d}}$.
\end{theorem}
\begin{proof}
    There are three claims in this theorem.
    \begin{claim}\label{clm:1}
        Every subgroup of a cyclic group is cyclic
    \end{claim}
    \begin{proof}[Proof of Claim]
      Let $G= \cyc{a}$ and suppose $H$ is a subgroup of $G$. We want to show that $H$ is cyclic. If $H = \Set{e}$ then we are done, so suppose $a^{t}\in H$ for some $t\neq 0$. If $t< 0$ then $a^{-t}\in H$ as well since $H$ is a (sub)group, so there is always an $a^{t}\in H$ such that $t>0$.
      
      Now let $m$ be the least positive integer such that $a^{m}\in H$ (we needed to show that the set of such integers is nonempty). Let $a^{k}\in H$; we wish to show that $a^{k} \in \cyc{a^{m}}$. Note that $k\geq m$ by minimality, so we can use the division algorithm to write $k = pm + r$ for integers $p,0\leq r<m$. Then
      \begin{equation*}
          a^{k} = a^{pm}a^{r} = a^{r}.
      \end{equation*}
      But $r<m$, so we must have $r=0$ (otherwise $m$ is not minimal). So $a^{k} = (a^{m})^{p}$, i.e. $a^{k} \in \cyc{a^{m}}$.
    \end{proof}
    \begin{claim}
      If $|a|=n$, then the order of any subgroup of $\cyc{a}$ is a divisor of $n$.
    \end{claim}
    \begin{proof}[Proof of Claim]
      By the above \autoref{clm:1}, if $H$ is a subgroup of $\cyc{a}$ we can write $H = \cyc{a^{m}}$ for some positive integer $m$. If $|H| = k$ then by \autoref{cor:ord_sub_div_ord_grp} we have $k \mid n$.
    \end{proof}
    \begin{claim}
      For each positive divisor $d$ of $|a|$, the group $\cyc{a}$ has exactly one subgroup of order $d$ --- namely $\cyc{a^{n/d}}$.
    \end{claim}
    \begin{proof}[Proof of Claim]
      If $d$ is any positive divisor of $n$ then by \autoref{thm:ord_cyc_sub},
      \begin{equation*}
        |\cyc{a^{n/d}}| = \frac{n}{\gcd(n,n/d)}
        = \frac{n}{n/d}
        = d.
        \end{equation*}
      Thus there is at least one subgroup of order $d$. Suppose another subgroup $H$ is of order $d$. By \autoref{clm:1} $H = \cyc{a^{m}}$ for some positive $m \mid n$ and $|\cyc{a^{m}}| = n/m = d$. So $m = n/d$, i.e. $H = \cyc{a^{n/d}}$.
    \end{proof}
\end{proof}
\begin{corollary}
  For each positive divisor $d$ of $n$, the set $\cyc{a^{n/d}}$ is the unique subgroup of $\Z_n$ of order $d$. Moreover, these are the only subgroups of $\Z_n$.
\end{corollary}
\begin{corollary}\label{cor:no_elm_euler_phi}
  The number of elements of order $d$ in a cyclic group of order $n$ is $\phi(d)$.
\end{corollary}
\begin{proof}
  By \autoref{thm:fund_cyc} the group has a unique subgroup of order $d$, which we call $\cyc{a}$. Note that an element $g\in G$ has order $d$ iff $|\cyc{g}| = d$ and so $|g| = d \iff \cyc{g} = \cyc{a}$. By \autoref{cor:generators_of_grp} an element $a^{k}$ generates $\cyc{a}$ iff $\gcd(k,d) = 1$. There are exactly $\phi(d)$ such elements.
\end{proof}
\begin{remark}
  The fundamental theorem can be used to exhaustively list all the subgroups of a finite cyclic group: they are exactly $\cyc{a^{d}}$ where $d$ is a divisor of $n = |a|$.
\end{remark}
\begin{corollary}
  In a finite group (not necessarily cyclic!), the number of elements of order $d$ is a multiple of $\phi(d)$.
\end{corollary}
\begin{proof}
  If a finite group has no elements of order $d$ then the statement is true, so let $a\in G$ with $|a| = d$. By \autoref{cor:no_elm_euler_phi} we know $\cyc{a}$ has $\phi(d)$ elements of order $d$. If all elements of order $d$ in $G$ are in $\cyc{a}$, then we are done, so suppose there is some $b\in G$ not in $\cyc{a}$. Then $\cyc{b}$ has $\phi(d)$ elements of order $d$ and we have found $2 \phi(d)$ elements of order $d$, provided that  $\cyc{a}$ and $\cyc{b}$ have no elements of order $d$ in common.

  But this must be the case, otherwise if $c$ is such an element then $\cyc{a} = \cyc{c} = \cyc{b}$ (recall that elements of order $d$ generate the cyclic subgroup they are members of). Continuing in this fashion we see that the number of elements of order $d$ in a finite group is a multiple of $\phi(d)$.
\end{proof}

\section{Permutation Groups}
\begin{definition}
    A \textit{permutation} of a set $A$ is a bijection $f:A\to A$. A \textit{permutation group} of a set $A$ is a set of permutations of $A$ that forms a group under function composition.
\end{definition}
\begin{definition}
  For a permutation $f$ of $\Set{1,\dots,n}$ we sometimes express $f$ in array form. So if $f(1) = 2$, $f(2) = 3$, $f(3) = 1$, and $f(4)=4$, we write
  \begin{equation*}
    f = \mat{1 & 2 & 3 & 4 \\ 2 & 3 & 1 & 4}.
  \end{equation*}
\end{definition}
\begin{definition}
  Let $A = \Set{1,\dots, n}$. The set of all permutations of $A$ is called the \textit{symmetric group of degree $n$} and is denoted by $S_n$. Clearly $|S_n| = n!$.
\end{definition}
\begin{theorem}
    The symmetric group $S_n$ is non-Abelian if $n\geq 3$.
\end{theorem}
\begin{definition}
  We can represent certain permutations $\alpha : \Set{1,\dots , n} \to \Set{ 1, \dots ,n }$ in cycle form. If $\alpha(1) = 2, \alpha(2) = 3, \dots , \alpha(n) = 1$, then we write
  \begin{equation*}
    \alpha = (123\dots n)
  \end{equation*}
\end{definition}
\begin{definition}
    Cycles of length 2 are often called \textit{transpositions}.
\end{definition}
\begin{theorem}
    Every permutation of a finite set can be written as a cycle or as a product of disjoint cycles.
\end{theorem}
\begin{proof}
  Note that in this case, the product refers to function composition. For a permutation $\alpha$ of $\Set{1,\dots,n}$ choose any member, say, $a_1$, and observe that
  \begin{equation*}
    (a_1,\alpha(a_1),\alpha^{2}(a_1),\dots)
  \end{equation*}
  is a finite cycle. Then simply choose an element $b_1$ not appearing in this cycle and continue the process, after which we obtain another cycle. These must be disjoint, otherwise we would have $\alpha^{i}(a_1) = \alpha^{j}(b_1)$ and so $b_1 = \alpha^{i-j}(a_1)$ which contradicts the way $b_1$ was chosen. Continuing in this manner we obtain a product of disjoint cycles.
\end{proof}

\begin{theorem}[Disjoint Cycles Commute]
  If the pair of cycles $\alpha = (a_1, \dots, a_m)$ and $\beta = (b_1, \dots, b_n)$ have no entries in common, then $\alpha \beta = \beta \alpha$.
\end{theorem}
\begin{theorem}
    The order of a permutation of a finite set written in disjoint cycle form is the least common multiple of the lengths of the cycles.
\end{theorem}
\begin{proof}
  Observe that a cycle of length $n$ has order $n$ (clearest seen with a diagram). Let $|\alpha| = m$ and $|\beta| = n$ and set $k:=\lcm (m,n)$. It follows from \autoref{thm:criterion_for_ai_eq_aj} that both $\alpha^{k}$ and $\beta^{k}$ are the identity permutation $\varepsilon$ and, since $\alpha$ and $\beta$ commute, $(\alpha \beta)^{k} = \alpha^{k} \beta^{k}$ is also the identity.

  Let $t = |\alpha \beta|$; we wish to show that $t = k$. By \autoref{cor:ak_eq_e_iff_ord_div_k} we know that $t \mid k$, so $k\geq t$. Now observe that since $\alpha$ and $\beta$ are disjoint, we know $\alpha^{t}$ fixes elements of $\beta$ and $\beta^{t}$ fixes elements of $\alpha$. Thus $\alpha^{t} \beta^{t}$ fixes $1$ through $n$ only when $\alpha^{t} = \beta^{t} = \varepsilon$. But this implies that $t$ is a multiple of both $m$ and $n$. Since $k= \lcm(m,n)$, we have $k\leq t$, and so we've shown that $k=t$, as desired.

  We've actually only proved the case for two disjoint cycles, but this can be readily extended to the general case by a quick induction.
\end{proof}
\begin{example}
    Determine all the orders of the $6!$ elements of $S_6$.
\end{example}
\begin{proof}[Solution]
    We simply write out the possible disjoint cycle structures of the elements of $S_6$. We have
    \begin{align*}
      (6) \\
      (5)(1) \\
      (4)(2) \\
      (4)(1)(1) \\
      (3)(3) \\
      (3)(2)(1) \\
      (3)(1)(1)(1) \\
      (2)(2)(2) \\
      (2)(2)(1)(1) \\
      (2)(1)(1)(1)(1) \\
      (1)(1)(1)(1)(1)(1).
   \end{align*}
   From the above theorem we see that the orders of the elements of $S_6$ are $7,5,4,3,6,2,1$.
\end{proof}
\begin{theorem}
  Every permutation in $S_n$, $n\geq 2$, is a product of $2$-cycles (or \textit{transpositions}).
\end{theorem}
\begin{proof}
  Observe that $(a_1a_2\dots a_n)$ is equal to $(a_1a_k)(a_1a_{k-1})\dots(a_1a_2)$. Note that the identity can be written as $(12)(12)$.
\end{proof}
\begin{remark}
    This is not the only way to write a permutation as a product of $2$-cycles. Observe that
    \begin{equation*}
      (12345) = (54)(53)(52)(51)
    \end{equation*}
    \begin{equation*}
      (12345) = (54)(52)(21)(25)(23)(13)
    \end{equation*}
    so even the number of 2-cycles may vary.
\end{remark}
There is one aspect of a decomposition that never varies. To prove this we will prove a preliminary lemma.
\begin{lemma}\label{lem:id_is_even}
    If $\varepsilon = \beta_1 \beta_2 \dots \beta_r$, where the $\beta_i$'s are $2$-cycles, then $r$ is even.
\end{lemma}
\begin{proof}
  The book's proof is rather convoluted. Here is a "standard" proof according to Gemini.

  Consider a polynomial in $n$ variables $x_{1},x_{2},\dots , x_{n}$ defined as the product of all differences $(x_i-x_j)$ where $i<j$:
  \begin{equation*}
    P(x_{1},x_{2},\dots,x_{n}) := \prod_{1\leq i< j \leq n} (x_i - x_j).
  \end{equation*}
  (The astute observer may recognize this as the determinant of a Vandermode matrix). For any permutation $\sigma \in S_n$ we define the action of $\sigma$ on the polynomial $P$ by permuting the indices of the variables:
  \begin{equation*}
    \sigma (P) := \prod_{1\leq i<j \leq n} (x_{\sigma(i)} - x_{\sigma(j)}).
  \end{equation*}
  Since the set of factors in $\sigma(P)$ is the same as in $P$ except perhaps for its sign, we have $\sigma(P) = \pm P$. Now consider the effect of a transposition, say, $\tau = (k,l)$ with $k<l$, on $P$.
  \begin{enumerate}

    \item The factor $(x_{k}-x_{l})$ becomes $(x_{l}-x_{k})$ and so its sign is flipped.
    \item For a fixed $m\neq k,l$ the factors $(x_m - x_k)$ and $(x_m - x_l)$ are a pair, possibly with inside terms ordered differently, come in pairs. Then $\tau$ merely swaps $x_{k}$ and $x_{l}$, leaving the sign unchanged.
    \item The rest of the factors are unaffected by $\tau$, leaving the sign unchanged.

  \end{enumerate}
  Thus a transposition $\tau$ has the effect $\tau(P) = - P$. We are given that
  \begin{equation*}
      \varepsilon = \beta_1 \beta_{2} \dots \beta_r
  \end{equation*}
  with each $\beta$ a transposition. Then applying this to the polynomial $P$:
  \begin{align*}
    P = \epsilon(P) &= (\beta_1 \beta_2 \dots \beta_r)(P) \\
    &= (-1)^{r}P
  \end{align*}
  and so $r$ must be even.
\end{proof}
\begin{remark}
    The argument may not seem so convincing since we're using a specific case to find a property of the general $S_n$ group. It may help to think of the counterfactual: if we could express the identity as a product of an odd number of transpositions, apply it to the Vandermode polynomial and we would get a contradiction.
\end{remark}
Now the main theorem is easily proven.
\begin{theorem}\label{thm:permutation_transposition_parity}
  If a permutation $\alpha$ can be expressed as a product of an even (odd) number of transpositions, then every decomposition of $\alpha$ into a product of transpositions must have an even (odd) number of transpositions. In symbols, if
  \begin{equation*}
    \alpha = \beta_{1} \beta_{2} \dots \beta_{r}\quad\text{and}\quad \alpha = \gamma_{1} \gamma_{2} \dots \gamma_{s}
  \end{equation*}
  where the $\beta$'s and $\gamma$'s are transpositions, then $r$ and $s$ have the same parity.

\end{theorem}
\begin{proof}
    Observe that $\beta_{1} \beta_{2} \dots \beta_{r}= \gamma_{1} \gamma_{2} \dots \gamma_{s}$ implies that
    \begin{align*}
      \varepsilon &= \gamma_{1} \gamma_2 \dots \gamma_s \beta^{-1}_r \dots \beta^{-1}_{2} \beta^{-1}_{1}  \\
      &= \gamma_{1} \gamma_{2} \dots \gamma_{s} \beta_{r} \dots \beta_{2} \beta_{1}
    \end{align*}
    since a transposition is its own inverse. Thus by \autoref{lem:id_is_even}, $s+r$ is even. It follows that $r$ and $s$ are both even or both odd.
\end{proof}

\begin{definition}
  A permutation that can be expressed as a product of an even (odd) number of transpositions is called an \textit{even} \textit{(odd)} permutation. 
\end{definition}
\begin{theorem}
    The set of even permutations in $S_n$ forms a subgroup of $S_n$.
\end{theorem}
\begin{proof}
  We use the One-Step Subgroup Test. We can write $\varepsilon = (12)(12)$, so it is in the set of even permutations. Let $\alpha$ and $\beta$ be even permutations. Note that $\beta^{-1}$ is even as well, so $\alpha \beta^{-1}$ must be even.
\end{proof}

\begin{definition}
   The group of even permutations of $n$ symbols is denoted by $A_{n}$ and is called the \textit{alternating group of degree $n$}.
\end{definition}
\begin{theorem}
    For $n\geq 2$, $A_n$ has order $n!/2$.
\end{theorem}
\begin{proof}
  Let $B_n$ denote the set of odd permutations in $S_n$. For an odd permutation $\alpha$, consider the function $f:B_n\to A_n$ defined by $f(\alpha) = (12) \alpha$. By cancellation properties we have $(12) \alpha = (12) \beta$ implies $\alpha = \beta$, so $f$ is an injection; in other words $|B_n| \leq |A_n|$.

  Similarly the function $g: A_n \to B_n$ defined by $g(\beta) = (12) \beta$ is an injection and so $|A_n|\leq |B_n|$. From this we conclude $|A_n| = |B_n|$ and since $A_n\cap B_n = \emptyset$, we obtain $|A_n| = n!/2$.
\end{proof}

\begin{remark}
  The name \textit{alternating} of $A_n$ comes from polynomials where transpositions change (alternate) its sign, exactly as in the proof for \autoref{lem:id_is_even}.
\end{remark}

\section{Isomorphisms}
\begin{definition}
  A homomorphism $\phi: G \to \overline{G}$ is a mapping that preserves the group operation. That is,
  \begin{equation*}
    \phi(ab) = \phi(a) \phi(b)\qquad \text{for all $a,b$ in $G$}.
  \end{equation*}
\end{definition}
There are names for special kinds of homomorphisms.
\begin{enumerate}

    \item An injective homomorphism is also called a \textit{monomorphism}
    \item A surjective homomorphism is also called an \textit{epimorphism}
    \item A bijective homomorphism is also called an \textit{isomorphism}
    \item An isomorphism from a group onto itself is also called an \textit{automorphism}

\end{enumerate}
\begin{definition}
    If there is an isomorphism $\phi: G\to \overline{G}$, we say that $G$ and $\overline{G}$ are \textit{isomorphic} and write $G\cong \overline{G}$.
\end{definition}
\begin{example}
  Let $G = \cyc{a}$. If $|a|= \infty$ then $G\cong \Z$ via the isomorphism $\phi(a^{k}) = k$. On the other hand if $|a|= n<\infty$ then $G\cong \Z_{n}$ via the isomorphism $\phi(a^{k}) = k \pmod{n}$.
\end{example}
\begin{theorem}[Properties of Isomorphisms on Elements]
  Let $\phi:G \to \overline{G}$ be an isomorphism. Then:
  \begin{enumerate}

    \item $\phi(e)$ is the identity of $\overline{G}$.
    \item $\phi(a^{n}) = [\phi(a)]^{n}$.
    \item $ab = ba$ iff $\phi(a) \phi(b) = \phi(b) \phi(a)$....
    \item $G = \cyc{a}$ iff $\overline{G} = \cyc{\phi(a)}$.
    \item $|a| = |\phi(a)|$.
    \item The equation $x^{k}=b$ has the same number of solutions in $G$ as does the equation $x^{k} = \phi(b)$ in $\overline{G}$.
    \item $G$ and $\overline{G}$ have the same number of elements of each order.

  \end{enumerate}
\end{theorem}
\begin{theorem}[Properties of Isomorphisms on Groups]
  Let $\phi:G \to \overline{G}$ be an isomorphism. Then:
  \begin{enumerate}

    \item $\phi^{-1}$ is an isomorphism from $\overline{G}$ onto $G$.
    \item $G$ is Abelian iff $\overline{G}$ is Abelian.
    \item $G$ is cyclic iff $\overline{G}$ is cyclic.
    \item If $K$ is a subgroup of $G$, then $\phi(K)$ is a subgroup of $\overline{G}$.
    \item $\phi(Z(G)) = Z(\overline{G})$

  \end{enumerate}
\end{theorem}

\begin{theorem}[Cayley's Theorem]
    Every finite group $G$ is isomorphic to a group of permutations. In particular,
    \begin{equation*}
      G \cong \Set{\pi_{g} \given g\in G}
    \end{equation*}
    Where $\pi_g(x):=gx $.
\end{theorem}
\begin{proof}
  We will show that $\phi : G \to \overline{G}$ defined by
  \begin{equation*}
    \phi(g) = \pi_g(x) \equiv gx
  \end{equation*}
  (i.e. $g\mapsto \pi_g$) is an isomorphism. Firstly, note that for any $g\in G$, $\pi_g$ is indeed a permutation (i.e. a bijection), as for all $x,y \in G$,
  \[
    \pi_g(x) = \pi_g(y) \implies gx=gy \implies x = y 
  \]
  and hence $\pi_g$ is an injection. And for any $x\in G$ we have $\pi_g(g^{-1}x) = x$, so $\pi_g$ is surjective.

  Now let us show that $\overline{G} = \Set{T_g \given g\in G}$ is a group under function composition:
  \begin{enumerate}

      \item Function composition is associative.
      \item Let $T_{g}\in \overline{G}$. Then for all $x$ we have $T_g T_e(x) = gex = gx = egx = T_{e} T_{g}(x) = T_{g}(x)$ and so $T_{e}$ is a proper identity.
      \item Let $T_{g} \in \overline{G}$. Then
        \begin{equation*}
          T_{g}T_{g^{-1}}(x) = gg^{-1}x = x =  g^{-1}gx = T_{g^{-1}}T_{g}(x) = T_{e}(x)
        \end{equation*}
        for all $x$ and hence each element has an inverse.

  \end{enumerate}

  We have shown that $\overline{G}$ is indeed a group of permutations. We now proceed to show that $\phi:G\to \overline{G}$ is an isomorphism. By construction, $\phi$ is onto. And $\phi$ is one-to-one since
  \begin{equation*}
    T_{g} = T_{h} \implies T_{g}(e) = T_{h}(e) \implies g = h.
  \end{equation*}
  We are left to show that $\phi$ is a homomorphism. Let $a,b\in G$. Then
  \begin{equation*}
    \phi(ab) = T_{ab} = T_{a}T_{b} = \phi(a) \phi(b).
  \end{equation*}
\end{proof}

\begin{definition}
  The group $\overline{G} = \Set{\pi_g \given g \in G}$ acting on the set $G$ in Cayley's Theorem is called the \textit{left regular representation of $G$}.
\end{definition}
\begin{remark}
  This $\overline{G}$ can be identified as a subgroup of $S_{|G|}$. Hence each $\pi_{g}$ can be represented in disjoint cycle form.
\end{remark}
\begin{example}
  The left regular representation for $U(12) = \{1,5,7,11\}$ can be obtained by writing the permutations of $U(12)$ in array form (much like a Cayley table):
    \begin{equation*}
      T_1 = \mat{1 & 5 & 7 & 11 \\ 1 & 5 & 7 & 11}, \quad 
      T_5 = \mat{1 & 5 & 7 & 11 \\ 5 & 1 & 11 & 7}
    \end{equation*}
    \begin{equation*}
      T_7 = \mat{1 & 5 & 7 & 11 \\ 7 & 11 & 1 & 5}, \quad 
      T_{11} = \mat{1 & 5 & 7 & 11 \\ 11 & 7 & 5 & 1}
    \end{equation*}
    In cycle form we obtain:
    \begin{equation*}
      T_1 = (1),\quad T_5 = (1,5)(7,11),\quad T_7 = (1,7)(5,11),\quad T_{11} = (1,11)(5,7)
    \end{equation*}
\end{example}
\begin{example}
    Let $G$ be a group of order $2n$ where $n$ is odd. Prove that if $G$ has an element of order $2$, then $G$ has a subgroup of order $n$.
\end{example}
\begin{proof}
    Let $a\in G$ with $|a|=2$. Viewing $G$ as a group of permutations, observe that since $\pi_a^{2} = e$, the permutation $\pi_a$ written in disjoint cycle form consists solely of transpositions. We cannot have a $1$-cycle, since that would imply $\pi_a$ sends an element to itself, which means $a=e$, a contradiction.

    Thus, $\pi_a$ consists of $n$ transpositions, which means it is an odd permutation. Define $A:= \Set{\sigma\in \overline{G} \given \text{ $\sigma$ is even}}$ and $B:= \Set{\tau\in \overline{G} \given \text{ $\tau$ is odd}}$, the sets of odd and even permutations in $\overline{G}$. Then the mapping $\sigma \mapsto \pi_a \sigma$ is an injection from $A$ to $B$ and $\tau \mapsto \pi_{a} \tau$ is an injection from $B$ to $A$. Therefore, $|A| = |B| = n$ and in fact $A$ is a subgroup of order $n$, as desired.
\end{proof}
\end{document}
